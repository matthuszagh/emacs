\documentclass{default}

\begin{document}

\tableofcontents
\hypersetup{linkcolor=red}

\chapter{Installation}\label{cha:installation}

I've fetched Emacs from the \href{https://github.com/emacs-mirror/emacs}{GitHub mirror
  repository}. The source code resides in \textasciitilde/developer/software/emacs. Installation
instructions reside in the source directory in the file named ``INSTALL''. I've chosen to use a
build directory named ``build''. From the build directory run configure with:

\begin{minted}{bash}
$ ./../configure --with-x-toolkit=gtk3 --with-mailutils \
                 --with-imagemagick --with-xwidgets
\end{minted}

I'm currently using version 26.1, by running:

\begin{minted}{bash}
$ git checkout emacs-26.1
\end{minted}

There is some issue with highlighting in version 27.050 (potentially font-lock?) that makes it very
annoying to write code. In general I should only use stable releases, which are specified on the
\href{https://www.gnu.org/software/emacs/}{Emacs website}.

\chapter{Package Setup}\label{cha:package-setup}

\section{Gnus}
\label{sec:gnus}

Gnus was a huge pain to setup. I finally got it working with the information at this
\href{https://eschulte.github.io/emacs-starter-kit/starter-kit-gnus-imap.html}{link}.

\section{Term}\label{sec:term}

Ansi term uses 8 colors for the terminal GUI. We can redefine these colors in an Emacs terminal by
customizing their values with \mintinline{text}{M-x customize-group RET term RET}. The colors must
also be configured in \mintinline{text}{~/.bashrc}.

\chapter{Functionality}
\label{cha:functionality}

\section{Code Tagging and Navigation}
\label{sec:code-tagging}



\section{Code Completion}
\label{sec:code-completion}

I'm currently using company with various backends for completion in different major modes. For C/C++
I'm using company-rtags. This potentially has the benefit of being aware of the current project
(e.g. completions for functions I've defined and custom headers) although this is
unconfirmed. However, it's a bit slow and some of the completions don't seem entirely
accurate. Another option could be company-ycmd. I believe this is faster although it may come at the
expense of being unaware of the environment. This should be verified.

\chapter{Structure}\label{cha:structure}


\chapter{Best Practices}
\label{cha:best-practices}

\href{https://www.gnu.org/software/emacs/manual/html_node/elisp/Key-Binding-Conventions.html}{This
  Emacs manual page} contains principles to follow when customizing key bindings. Basically,
\mintinline{text}{C-c <letter>} (but not \mintinline{text}{C-c} followed by another control
character) as well as \mintinline{text}{<f5>} through \mintinline{text}{<f9>} are free for users to
define how they wish.


\chapter{To-Do}\label{cha:to-do}

\section{Customize Mode Line}
\label{sec:customize-mode-line}

I can customize mode-line-format on a per-mode basis (see
\href{https://emacs.stackexchange.com/questions/13652/how-to-customize-mode-line-format}{this
  StackExchange question}). Also, see the
\href{https://www.gnu.org/software/emacs/manual/html_node/elisp/Mode-Line-Variables.html#Mode-Line-Variables}{Emacs
  documentation} on the subject.

\section{Disable cursor in pdf-view-mode}

I'd like to disable the cursor when viewing a PDF. It's distracting and provides no value. I've
tried these two additions to \mintinline{text}{init.el} to no avail:

\begin{minted}{elisp}
(add-hook 'pdf-view-mode-hook
          (lambda ()
            (make-variable-buffer-local 'cursor-type)
            (setq cursor-type nil)))
(add-hook 'post-command-hook
          (lambda ()
            (setq cursor-type (if pdf-view-mode t 'nil))))
\end{minted}

These are the links I've found that address related issues:

\href{https://emacs.stackexchange.com/questions/392/how-to-change-the-cursor-type-and-color}{How to
  change the cursor type and color?}

\href{https://www.gnu.org/software/emacs/manual/html_node/elisp/Cursor-Parameters.html}{Cursor
  parameters}

\href{https://emacs.stackexchange.com/questions/44650/how-can-i-make-the-cursor-change-to-block-in-overwrite-mode?rq=1}{How
  can I make the cursor change to block in overwrite mode?}

\href{https://www.emacswiki.org/emacs/ChangingCursorDynamically}{Changing Cursor Dynamically}

\href{https://www.gnu.org/software/emacs/manual/html_node/emacs/Cursor-Display.html}{Cursor Display}

\section{Get much better at Gnus}

\section{Company Backends}

There are various company backends that don't work. At the very least company-reftex and
company-auctex aren't working. Based on info from the company-backends variable, it sounds like only
one backend can be used at a time. This might be the issue.

\section{load-file from Emacs}

This causes cursor color to change and potentially other behavior as well.

\section{align-current AUCTeX}

align-current works well for tables with limited text. However, it propagates line breaks and so
takes up a needless number of lines for long text sections that have to be manually adjusted. Also,
I'd like to be able to run this globally for a whole buffer. The most similar action is align-entire
but this aligns all tables to one another; I'd like aligning to be done on a per-table basis.

\section{Shortcut keys not working in Auctex}

I have to manually load the init file after emacsclient -nc in order for C-i keybindings to
work. This should work off the bat, and in general its weird that loading the init file has any
effect at all.

\section{align uses 2 spaces in Emacs}

The align environment seems to use 2 spaces in Auctex instead of the 8 used by everything else. It
should be 8.

\section{C++ company completions are not always accurate}

For instance, \mintinline{text}{find_last_not_of} has incorrect completions. The old configuration
did provide correct completions.

\section{When Emacs opens it is not completely maximized}

The right side of the frame is not quite right.

\section{Single quote delimeter not working}

Pressing \' twice causes three single quotes to be inserted.

\section{Command that closes all buffers for which underlying file was deleted}

Probably better, just do this automatically. One worry is if file was accidentally deleted and
buffer allows you to keep the file alive. Still, existing buffers shouldn't be used as a sort of
backup.

\section{Window recenters in term mode}

Often, when I type a character in term mode the window recenters instead of just leaving the cursor
position where it is. I want it to just leave it there. This seems to relate to scrolling,
specifically scroll-conservatively,
etc. \href{https://www.gnu.org/software/emacs/manual/html_node/emacs/Auto-Scrolling.html}{This link}
is useful. \href{https://emacs.stackexchange.com/a/21466/20317}{This SO answer} describes how window
points work and may also be useful. This appears to be caused by an interaction with helm. Test this
by fully disabling helm and then testing.

\end{document}
%%% Local Variables:
%%% mode: latex
%%% TeX-master: t
%%% End:
