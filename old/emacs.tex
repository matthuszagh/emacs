\documentclass{default}

\begin{document}

\tableofcontents
\hypersetup{linkcolor=red}

\chapter{Installation}\label{cha:installation}

I've fetched Emacs from the \href{https://github.com/emacs-mirror/emacs}{GitHub mirror
  repository}. The source code resides in \textasciitilde/developer/software/emacs. Installation
instructions reside in the source directory in the file named ``INSTALL''. I've chosen to use a
build directory named ``build''. From the build directory run configure with:

\begin{minted}{bash}
$ ./../configure --with-x-toolkit=gtk3 --with-mailutils \
                 --with-imagemagick --with-xwidgets
\end{minted}

I'm currently using version 26.1, by running:

\begin{minted}{bash}
$ git checkout emacs-26.1
\end{minted}

There is some issue with highlighting in version 27.050 (potentially font-lock?) that makes it very
annoying to write code. In general I should only use stable releases, which are specified on the
\href{https://www.gnu.org/software/emacs/}{Emacs website}.

\chapter{Package Setup}\label{cha:package-setup}

\section{Gnus}
\label{sec:gnus}

Gnus was a huge pain to setup. I finally got it working with the information at this
\href{https://eschulte.github.io/emacs-starter-kit/starter-kit-gnus-imap.html}{link}.

\section{Term}\label{sec:term}

Ansi term uses 8 colors for the terminal GUI. We can redefine these colors in an Emacs terminal by
customizing their values with \mintinline{text}{M-x customize-group RET term RET}. The colors must
also be configured in \mintinline{text}{~/.bashrc}.

\chapter{Functionality}
\label{cha:functionality}

\section{Code Tagging and Navigation}
\label{sec:code-tagging}


\chapter{Structure}\label{cha:structure}


\chapter{Best Practices}
\label{cha:best-practices}

\href{https://www.gnu.org/software/emacs/manual/html_node/elisp/Key-Binding-Conventions.html}{This
  Emacs manual page} contains principles to follow when customizing key bindings. Basically,
\mintinline{text}{C-c <letter>} (but not \mintinline{text}{C-c} followed by another control
character) as well as \mintinline{text}{<f5>} through \mintinline{text}{<f9>} are free for users to
define how they wish.


\chapter{To-Do}\label{cha:to-do}

\section{Customize Mode Line}
\label{sec:customize-mode-line}

I can customize mode-line-format on a per-mode basis (see
\href{https://emacs.stackexchange.com/questions/13652/how-to-customize-mode-line-format}{this
  StackExchange question}). Also, see the
\href{https://www.gnu.org/software/emacs/manual/html_node/elisp/Mode-Line-Variables.html#Mode-Line-Variables}{Emacs
  documentation} on the subject.

\section{Disable cursor in pdf-view-mode}

I'd like to disable the cursor when viewing a PDF. It's distracting and provides no value. I've
tried these two additions to \mintinline{text}{init.el} to no avail:

\begin{minted}{elisp}
(add-hook 'pdf-view-mode-hook
          (lambda ()
            (make-variable-buffer-local 'cursor-type)
            (setq cursor-type nil)))
(add-hook 'post-command-hook
          (lambda ()
            (setq cursor-type (if pdf-view-mode t 'nil))))
\end{minted}

These are the links I've found that address related issues:

\href{https://emacs.stackexchange.com/questions/392/how-to-change-the-cursor-type-and-color}{How to
  change the cursor type and color?}

\href{https://www.gnu.org/software/emacs/manual/html_node/elisp/Cursor-Parameters.html}{Cursor
  parameters}

\href{https://emacs.stackexchange.com/questions/44650/how-can-i-make-the-cursor-change-to-block-in-overwrite-mode?rq=1}{How
  can I make the cursor change to block in overwrite mode?}

\href{https://www.emacswiki.org/emacs/ChangingCursorDynamically}{Changing Cursor Dynamically}

\href{https://www.gnu.org/software/emacs/manual/html_node/emacs/Cursor-Display.html}{Cursor Display}

Instead of trying to get rid of the cursor, maybe just make its color the same as the
background. Check out these links:
\href{https://emacs.stackexchange.com/questions/7281/how-to-modify-face-for-a-specific-buffer}{se}
and \href{https://www.gnu.org/software/emacs/manual/html_node/elisp/Face-Remapping.html}{manual}.

\section{Get much better at Gnus}

\section{Flycheck doesn't work with C}

I'm getting an error that the flag -std=c++17 doesn't work with
C\@. \href{https://github.com/alexmurray/flycheck-clang-analyzer/issues/6}{This error seems somewhat
similar}.

\section{load-file from Emacs}

This causes cursor color to change and potentially other behavior as well.

\section{align-current AUCTeX}

align-current works well for tables with limited text. However, it propagates line breaks and so
takes up a needless number of lines for long text sections that have to be manually adjusted. Also,
I'd like to be able to run this globally for a whole buffer. The most similar action is align-entire
but this aligns all tables to one another; I'd like aligning to be done on a per-table basis.

\section{Consider switching to sane-term from multi-term}

\href{https://github.com/adamrt/sane-term}{sane-term} seems like it might be a better alternative to multi-term.

\section{Single quote delimeter not working}

Pressing \' twice causes three single quotes to be inserted.

\section{Command that closes all buffers for which underlying file was deleted}

Probably better, just do this automatically. One worry is if file was accidentally deleted and
buffer allows you to keep the file alive. Still, existing buffers shouldn't be used as a sort of
backup.

\section{Magit bug missing headers}

I'm getting the error: ``BUG: missing headers nil'' when using magit with ghub. This is a known
issue and is tracked \href{https://github.com/magit/ghub/issues/81}{here}. I guess the solution is
to just wait for someone to fix it there.

\section{Verilog mode escape}

When in evil insert mode and the cursor position is after a reg or wire, pressing escape tries to
vectorize the reg/wire instead of entering evil normal state.

\section{Verilog code tagging}

Verilog mode should have code tagging and navigation. See
\href{https://scripter.co/ctags-systemverilog-and-emacs/}{this link} for ideas on code tagging.

\section{Better Code Tagging}

The problem with RTags is that it only works for C-based languages. gtags should work for many
others. See \href{https://tuhdo.github.io/c-ide.html}{this link}. It probably makes sense to use the
\href{https://github.com/leoliu/ggtags}{ggtags} interface. Also use the
\href{https://github.com/syohex/emacs-helm-gtags/}{helm interface}. This may also be the wrong
question, but maybe it's also possible to use rtags as a backend for a common gtags frontend?

\section{Create Add-On that allows you to edit Anki HTML in Emacs}

Adapt \href{https://github.com/louietan/anki-editor}{this} for your needs. That add-on should
contain nearly all of the required functionality.

\section{Get XWidget Webkit working}

See \href{https://emacsnotes.wordpress.com/2018/08/18/why-a-minimal-browser-when-there-is-a-full-featured-one-introducingxwidget-webkit-a-state-of-the-art-browser-for-your-modern-emacs/}{this}.

\section{Get LaTeX SVGs working in EWW}

EWW is unable to render latex/mathjax directly and so renders the resulting svg instead. This
appears as dark text on a dark background. See
\href{https://emacs.stackexchange.com/questions/3622/use-a-different-color-theme-for-eww-buffers}{this
  post}.

\section{Improve org-books}

\begin{enumerate}
\item Query user for file when running org-capture and base title on that.
\item Make table of contents entries links to the corresponding page in the pdf.
\item Write a program to extract a table of contents from pdf. Run this automatically when running
  org-capture.
\item Bind org-capture to a keybinding.
\item Write scripts to get info from Goodreads and Amazon (Goodreads API
  \href{https://www.goodreads.com/api}{here}).
\item Add description property so that you can make comments on the quality (e.g. ``Definitive book
  on GR'').
\item Potentially use Wikipedia categories to categorize books in list and in directories.
\end{enumerate}

\section{Org Capture Extension}

Consider using \href{https://github.com/sprig/org-capture-extension}{org-capture-extension} which
allows you to run org-capture from a browser.

\section{Git Bug}

Consider using \href{https://github.com/MichaelMure/git-bug}{git-bug} instead of this file to track
issues and improvements.

\section{Eyebrowse}

Use \href{https://github.com/wasamasa/eyebrowse}{eyebrowse} to easily navigate between Emacs workstations.

\section{Setup BBDB or other contact list for GNUs}

bbdb keeps a contact list that can be used in gnus. Set this up, or look for an alternative
(potentially one that does not require manual additions). Note that there is also a company backend
for this, company-bbdb.

\section{Company specifies backend for each completion}

It would be nice to be able to show backend used for each completion with company. This could be a
configurable setting in company, for instance. Maybe look at company-box for a possible way to
implement this.

\section{Implement method to see what is causing lag in Emacs}

It would be really nice to be able to see what is causing Emacs to lag in certain contexts. This
could be enabled by calling a sort of record function and then disabling it when done. When the
record function is toggled off, display a list of functions called and how much collective time they
took, ordered by that time.

\section{Create a new gdb many windows mode that spans 2 screens}

This should also have a function to restore windows. It should delegate most responsibility to GDB\@.

\section{Change company completion face}

The red text face used for company completion is ugly. Find a thematic alternative.

\section{Fix YCMD no flags error}

YCMD sometimes complains and repeatedly throws a flags missing error. Figure out what causes this
(look at the YCMD code), then figure out how to solve it. This probably involves reconfiguration the
ycm extra conf file.

\section{Keep company-box?}

Consider the functionality of company-box and potentially get rid of it.

\section{Completion drop-down vs fill-in}

Sometimes completions provide a drop-down and sometimes it doesn't and instead presents a sole
completion as a differently colored rest of the word. What logic regulates this? Is it when only a
single completion is available? Is this disadvantageous for a sole completion when the dropdown
could also provide documentation for the completion?

\section{Improve GDB behavior for opening windows}

The current GDB split window behavior isn't great. It splits up existing windows into overly small
sizes. I'm not exactly sure what I want here, but I'd like a strong priority given to the gdb buffer
and the source buffer. In any event these two links should be helpful:
\href{https://emacs.stackexchange.com/questions/38945/m-x-gdb-dont-create-new-frames}{this} and
\href{https://www.gnu.org/software/emacs/manual/html_node/elisp/Display-Action-Functions.html}{this}.

\section{Enable ability in term mode when at command line to insert at point instead of going to
  end}

This would allow me to navigate in normal mode and then insert like normal.

\begin{minted}{lisp}
  (evil-collection-define-key 'normal 'term-mode-map (kbd "i")
    (lambda ()
      (interactive)
      (if (eq (line-number-at-pos)
              (+ (evil-count-lines (point-min) (point-max)) 1))
          (let* ((col-str (what-cursor-position))
                 (col (string-to-number
                       (when (string-match ".*column=" col-str)
                         (replace-match "" nil nil col-str)))))
            (progn (evil-insert-state)
                   (term-send-end)
                   (term-show-maximum-output)
                   (let* ((col-new-str (what-cursor-position))
                          (col-new (string-to-number
                                    (when (string-match ".*column=" col-new-str)
                                      (replace-match "" nil nil col-new-str)))))
                     (while (> col-new col)
                       (progn (execute-kbd-macro (kbd "C-b"))
                              (message (format "%d %d" col col-new))
                              (setq col-new (1- col-new)))))))
        (progn (evil-insert-state)
               (term-show-maximum-output)))))
\end{minted}

This doesn't work because edits made to the buffer line in term line mode are not affected in
term-char mode. This would take a lot more work.

\section{Prevent sudo directories from staying in recentf list}

We can probably configure this with tramp, see \href{https://www.gnu.org/software/tramp/tramp-emacs.html}{manual}.

\section{Store source URL of book and write function to detect if changes have occurred}

Use a hash function for this. Hopefully there's a way to do this without redownloading all files
each time.

\section{Get helm-make working}

See \href{https://github.com/abo-abo/helm-make}{here}.

\section{Code completion should ignore case}

Setup company in code completion modes to ignore entered case and fill in the correct case
regardless.

\section{Setup rr debugger with gdb interface}

See this
\href{https://emacs.stackexchange.com/questions/20056/is-it-possible-to-use-mozillas-rr-with-gdb-multi-window?rq=1}{link}
on setting it up. See this other
\href{http://fitzgeraldnick.com/2015/11/02/back-to-the-futurre.html}{link} on what it can do.

\section{Function to determine an installed packages dependencies}

Some package is using magit-popup which is causing a warning each time magit is used. Figure out
which package. More generally, write a function to figure out which packages use a given
dependency. Additionally, have another function to list all dependencies for a given input.

\section{Fix info manual building with straight}

Most info manuals aren't built.

\section{Make c-style-alist style options buffer local from clang-format}

c-style-alist specifies a number of style options that should be set in the same way as the other
style options from clang-format.

\section{Setup smartparens}

Use smartparens instead of other packages.

\section{flycheck clang tidy automatically find compile commands}

Find compile\_commands.json recursively in current project directory, rather than needing to specify
it.

\section{Init file unit testing}

I'd like to have some sort of testing for the init file to ensure that all configurations are
actually implemented and other changes do not affect them. The builtin ERT framework might be able
to achieve this. Also use auto-compile, which ensures byte-compiled files are up-to-date with the
source versions.

\section{PDFs open in both buffers sometimes}

This seems to happen when a PDF file is first opened. Investigate what happens when I disable the
bookmarking functionality. I may no longer need bookmarking anyway with org tracking.

\section{Ensure evil collection bindings are working}

Evil collection defines a framework for vim-like bindings that make a lot of sense. Become familiar
with these and make sure they're working. Finally, it would be great to set up testing for them.

\section{pdf tools fails to define function}

I get this error at startup. File mode specification error: (error Autoloading file /home/matt/.emacs.d/straight/build/pdf-tools/pdf-tools.elc failed to define function pdf-tools)

\section{jedi core installs each startup}

I get this message at each startup. Is this right? Running: pip install --upgrade /home/matt/.emacs.d/straight/build/jedi-core/...Done

\section{show links in pdf}

Implement a feature similar to Vimium's f key, which shows keys associated with each link on a
page. The links can be followed by typing the corresponding key. This would be useful in PDF mode so
the cursor does not need to be used.

\section{Org align contents of lists}

Org does not currently left-align the content of a list item.

\end{document}
%%% Local Variables:
%%% mode: latex
%%% TeX-master: t
%%% End:
